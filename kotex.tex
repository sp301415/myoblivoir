\usepackage[hangul, nonfrench, finemath]{kotex}
\kscntformat{section}{}{}

\usepackage{iftex}
\ifpdftex
    \input glyphtounicode
    \pdfgentounicode=1
\fi

\ifLuaTeX
    \hangulpunctuations=0

    % Taken from luatex-ko doc.
    \setmainhangulfont{Noto Serif CJK KR}[
        Scale=0.98,
        AutoFakeSlant,
        Script=Hangul,
        Language=Korean,
        BoldFont=* Bold,
        Expansion,
    ]
    \setsanshangulfont{Noto Sans CJK KR}[
        Scale=0.98,
        AutoFakeSlant,
        Script=Hangul,
        Language=Korean,
        BoldFont=* Bold,
        Expansion,
    ]
    \setmonohangulfont{Noto Sans Mono CJK KR}[
        Scale=0.98,
        AutoFakeSlant,
        Script=Hangul,
        Language=Korean,
        BoldFont=* Bold,
        Expansion,
    ]
    \setmathhangulfont{Noto Sans CJK KR}[
        Script=Hangul,
        Language=Korean,
        SizeFeatures={
                {Size=-6,  Font=* Medium},
                {Size=6-9, Font=* Regular},
                {Size=9-,  Font=* DemiLight},
            },
    ]
\fi
