\usepackage[hangul]{kotex}

\usepackage{iftex}
\ifPDFTeX
\input glyphtounicode
\pdfgentounicode=1
\fi

\ifLuaTeX
% Taken from luatex-ko doc.
\setmainhangulfont{Noto Serif CJK KR}[
  AutoFakeSlant,
  Script=Hangul,
  Language=Korean,
  BoldFont=* Bold,
  Protrusion,
  Expansion,
]
\setsanshangulfont{Noto Sans CJK KR}[
  AutoFakeSlant,
  Script=Hangul,
  Language=Korean,
  BoldFont=* Bold,
  Protrusion,
  Expansion,
]
\setmonohangulfont{Noto Sans Mono CJK KR}[
  AutoFakeSlant,
  Script=Hangul,
  Language=Korean,
  BoldFont=* Bold,
  Protrusion,
  Expansion,
]
\setmathhangulfont{Noto Sans CJK KR}[
  Script=Hangul,
  Language=Korean,
  SizeFeatures={
    {Size=-6,  Font=* Medium},
    {Size=6-9, Font=* Regular},
    {Size=9-,  Font=* DemiLight},
  },
]
\fi
